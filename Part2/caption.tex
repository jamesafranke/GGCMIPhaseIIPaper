Illustration of the effects of factors affecting yields in more realistic climate scenarios. 
Figures shows emulated yield changes for maize on currently cultivated land under RCP8.5 (relative to 1980-2010 mean) for three representative crop models, with changes to T only (left), to T and P (center), and to T, P, and CO2 (right)
 Circles are emulated decadal yields from 2010-2100 in scenarios from 29 CMIP0-5 climate models, i.e.\ 261 total, with x-axis the mean T shift over all grid cells where maize are grown (unweighted by within-cell cultivated area). Rug plot at bottom shows range of final (mean 2090-2100) temperatures across models
 Bold lines are the emulated values over uniform T shifts. Open squares in \textit{Left} panel are GGCMI Phase II simulated values for each T level (with CWN at baseline).  Emulations capture simulated behavior well (compare squares to lines), with the exception of PROMET at extreme temperature change. (See also Figure \ref{XX}.)
Mean yields are very similar for scenarios with a uniform temperature shift and those with more realistic temperature change patterns (compare lines to circles) 
\textit{Center}: adding in projected precipitation changes depresses yields slightly and increases spread between projections for a given temperature change. \textit{Right:} adding in CO$_2$ changes produces very different responses across models. CO$_2$ fertilization is small in pDSSAT, moderate in LPJmL, and very large in PROMET. 
Emulation uncertainty is small compared to the differences across climate and crop models.

simplify legend and put it in top-to-bottom order. Can go in two panels or just in left one, in which case rug plot goes in the center panel
(gold): PROMET
(blue) LPJmL
(black) pDSSAT
(circle) emulated RCP8.5
(line) emulated uniform T
(square) simulated uniform T



